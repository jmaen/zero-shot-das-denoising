\chapter{Background}\label{ch:background}

This chapter provides the necessary background for understanding unsupervised denoising. % TODO rephrase
We begin by defining the general denoising problem and discussing its inherent challenges.
We then present key deep learning concepts and techniques that are used in the context of denoising.
Finally, we introduce Digital Acoustic Sensing (DAS) as a real-world application and highlight the unique difficulties 
it presents.

\section{Denoising}

In general, denoising refers to the process of recovering a clean signal from a noisy observation.
Formally, it can be described by the inverse problem 
\begin{equation}
    y = x + n
\end{equation}
where $y$ is the noisy observation, $x$ is the underlying clean signal and $n$ represents some form of noise, for
example Gaussian noise $n \sim \normal$.
Denoising is an inherently ill-posed problem, as the noise and, in many cases, the noise distribution are unknown.
Therefore, additional assumptions about the solution are essential for solving the denoising problem. 
This process is known as regularization and typically involves imposing certain constraints on the solution space to 
favor more natural solutions~\cite{XXX}. 
The choice of regularizer depends on the specific problem setting and the type of data involved.

\section{Deep Learning}

Deep learning is a subfield of machine learning that utilizes deep neural networks to learn complex patterns from data. 
Over the past decade, deep learning has established itself as the state-of-the-art approach for a wide range of problems
across various different fields.

\subsection{Deep Neural Networks}

In its most basic form, a (fully-connected) neural network consists of neurons organized in layers, where each neuron
applies a linear transformation followed by a non-linear activation function.
The output of a single neuron is given by
\begin{equation}
    y = \varphi(\mathbf{w}^T\mathbf{x} + b)
\end{equation}
for an input $\mathbf{x}$, a weight vector $\mathbf{w}$, a bias value $b$ and an activation function $\varphi$. 
The activation function is needed in order to avoid the network collapsing into a single linear transformation. 
The outputs of each layer are then passed as inputs to the next layer. 
Therefore, a neural network can be described as a function $f_{\theta}: \mathcal{X} \rightarrow \mathcal{Y}$ 
parameterized by $\theta$, where $\theta$ represents the weights and biases across all layers~\cite{DeepLearning}.

In order to optimize these parameters, a loss function $\mathcal{L}: \mathcal{Y} \times \mathcal{Y} \rightarrow \R$ is 
defined, which measures the difference between the predicted output and the target value. 
Now, the gradient of the loss function with respect to the parameters, 
$\nabla_\theta \mathcal{L} = \frac{\partial \mathcal{L}}{\partial \theta}$, represents the direction of steepest ascent.
Therefore, by moving the parameters in the opposite direction of the gradient, the loss function can be minimized.
Typically, the gradient is not calculated for a single data point or for the whole dataset, but instead for a
small subset of the dataset, which is why this approach is referred to as (mini-batch) gradient descent.
Backpropagation~\cite{Backpropagation} is used to efficiently compute the gradient by making use of the
chain rule, enabling fast optimization.

While traditionally neural networks only consisted of a few layers and required hand-crafted features to work 
effectively, advances in computing power allow modern architectures to automate feature extraction by using additional 
layers, hence the term \textit{deep} neural network.

\subsection{Convolutional Neural Networks}

Convolutional neural networks (CNNs)~\cite{CNN} are a specific type of neural network that learns features using
kernels.
Prior to the rise of deep learning, such kernels were designed manually for various computer vision tasks, for example
the Sobel kernel~\cite{Sobel} used for edge detection.
In CNNs, these kernels are automatically learned from data. In contrast to fully-connected layers, the output of a
convolutional layer is obtained by convolution with one or multiple kernels, replacing the matrix multiplication.
For a kernel $\mathbf{K} \in \R^{n \times m}$, the convolution is defined as
\begin{equation}
    (\mathbf{X} \ast \mathbf{K})_{i,j} = \sum_{n}\sum_{m} X_{i+n,j+m} \cdot K_{n,m}.
\end{equation}
The output of the convolution is then passed through a non-linear activation function, just like in fully-connected
layers.
CNNs provide two main advantages: First, since the weights are shared, convolutional layers drastically reduce the
number of parameters compared to fully-connected layers. 
Second, convolutions are translationally equivariant, meaning that local patterns in the input can be recognized 
regardless of their position, which makes CNNs very suitable for image data~\cite{DeepLearning}.

\subsection{Normalization}

\newcommand{\cube}[2]{
    % background
    \fill[gray!10, opacity=0.5] (0,0,\a) -- (\a,0,\a) -- (\a,\a,\a) -- (0,\a,\a) -- cycle;
    \fill[gray!10, opacity=0.5] (0,0,0) -- (\a,0,0) -- (\a,0,\a) -- (0,0,\a) -- cycle;
    \fill[gray!10, opacity=0.5] (0,0,0) -- (0,\a,0) -- (0,\a,\a) -- (0,0,\a) -- cycle;

    % edges
    \draw[thick] (0,#1,0) -- (0,0,0) -- (#1,0,0);
    \draw[thick] (0,#1,0) -- (0,#1,#1);
    \draw[thick] (0,0,0) -- (0,0,#1);
    \draw[thick] (#1,0,0) -- (#1,0,#1);
    \draw[thick] (0,#1,#1) -- (0,0,#1) -- (#1,0,#1) -- (#1,#1,#1) -- cycle;

    % grid
    \foreach \i in {1,2,...,#1} {
        \draw[very thin] (0,\i,0) -- (0,\i,#1);
        \draw[very thin] (0,0,\i) -- (0,#1,\i);

        \draw[very thin] (\i,0,0) -- (\i,0,#1);
        \draw[very thin] (0,0,\i) -- (#1,0,\i);

        \draw[very thin] (\i,0,#1) -- (\i,#1,#1);
        \draw[very thin] (0,\i,#1) -- (#1,\i,#1);
    }

    % labels
    \node[rotate=90] at (0,#1,#1/2) [above] {$H,W$};
    \node at (0,#1/2,0) [below left] {$C$};
    \node at (#1/2,0,0) [below right] {$N$};

    % name
    \node at (#1,#1,#1+1) {#2};
}

\begin{figure}
    \centering

    \tdplotsetmaincoords{60}{-45}
    \def\a{4}
    \begin{tikzpicture}[tdplot_main_coords, scale=0.5]
        \fill[Maroon] (0,0,0) -- (\a,0,0) -- (\a,0,\a) -- (0,0,\a) -- cycle;
        \fill[Maroon] (0,0,0) -- (0,1,0) -- (0,1,\a) -- (0,0,\a) -- cycle;
        \fill[Maroon] (0,0,\a) -- (\a,0,\a) -- (\a,1,\a) -- (0,1,\a) -- cycle;

        \cube{\a}{Batch Norm}
    \end{tikzpicture}
    \hfill
    \begin{tikzpicture}[tdplot_main_coords, scale=0.5]
        \fill[Maroon] (0,0,0) -- (0,\a,0) -- (0,\a,\a) -- (0,0,\a) -- cycle;
        \fill[Maroon] (0,0,0) -- (1,0,0) -- (1,0,\a) -- (0,0,\a) -- cycle;
        \fill[Maroon] (0,0,\a) -- (0,\a,\a) -- (1,\a,\a) -- (1,0,\a) -- cycle;

        \cube{\a}{Layer Norm}
    \end{tikzpicture}
    \hfill
    \begin{tikzpicture}[tdplot_main_coords, scale=0.5]
        \fill[Maroon] (0,0,0) -- (0,1,0) -- (0,1,\a) -- (0,0,\a) -- cycle;
        \fill[Maroon] (0,0,0) -- (1,0,0) -- (1,0,\a) -- (0,0,\a) -- cycle;
        \fill[Maroon] (0,0,\a) -- (0,1,\a) -- (1,1,\a) -- (1,0,\a) -- cycle;

        \cube{\a}{Instance Norm}
    \end{tikzpicture}
    \hfill
    \begin{tikzpicture}[tdplot_main_coords, scale=0.5]
        \fill[Maroon] (0,0,0) -- (0,\a/2,0) -- (0,\a/2,\a) -- (0,0,\a) -- cycle;
        \fill[Maroon] (0,0,0) -- (1,0,0) -- (1,0,\a) -- (0,0,\a) -- cycle;
        \fill[Maroon] (0,0,\a) -- (0,\a/2,\a) -- (1,\a/2,\a) -- (1,0,\a) -- cycle;

        \cube{\a}{Group Norm}
    \end{tikzpicture}

    \caption{
        Different normalization techniques. $N$ is the batch dimension, $C$ is the channel dimension and $H$ and $W$ are
        the spatial dimensions of a 4D tensor. The input is normalized across the dimensions highlighted in red.
        Figure adapted from~\cite{GroupNorm}.
    }
    \label{fig:normalization}
\end{figure}

During the training process, the inputs of each layer change with each iteration as the parameters are optimized.
That slows down the training because now each layer has to adapt to the new distribution of its inputs. This process is
often referred to as internal covariate shift.
To counteract this issue, Ioffe \etal proposed Batch Normalization (BN)~\cite{BatchNorm}.
The idea behind BN is to normalize the inputs across the whole mini-batch and their spatial dimensions.
The normalized input for a channel $i$ is given by
\begin{equation}
    \hat{x}^{(i)} = \frac{x^{(i)} - \mu^{(i)}}{\sigma^{(i)}},
\end{equation} 
where $\mu^{(i)}$ and $\sigma^{(i)}$ are the per-channel mean and standard deviation of the mini-batch, respectively.  
In order to allow the model learn the identity if that is the optimal transformation, two additional learnable
parameters, $\gamma$ and $\beta$, are introduced. The output of the BN layer is then defined as
\begin{equation}
    y^{(i)} = \gamma^{(i)}\hat{x}^{(i)} + \beta^{(i)}.
\end{equation}
While BN is widely used, there exist various similar normalization techniques~\cite{LayerNorm, InstanceNorm, GroupNorm}
mainly differing in the dimensions across which they are applied.
A selection of them is visualized in Figure~\ref{fig:normalization}.


\subsection{Attention Mechanisms}

In neural networks, some input features are typically more important than others.
An attention mechanism helps the network to focus on (\textit{attend to}) the most relevant parts of the
input, rather than processing all inputs equally.
This works by dynamically reweighting the features based on their importance~\cite{Attention}.
While attention is often associated with natural language processing (NLP), especially since the introduction of the
Transformer architecture~\cite{Transformer} where it is the underlying key principle, it also has applications beyond
NLP.\@
In computer vision, for example, it can help CNNs by reweighting feature channels~\cite{SqueezeExcite} or
highlighting important spatial regions.

\section{Digital Acoustic Sensing}
