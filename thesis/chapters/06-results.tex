\chapter{Results}\label{ch:results}

This chapter presents the experimental results of different DIP-based denoising approaches.
We first evaluate these methods on standard image datasets, providing a controlled setting for method comparison.
Then, we apply these approaches to DAS data demonstrating its performance in a more complex real-world scenario.

\section{Image Data}

Since clean reference samples are not available for DAS data, we begin by conducting experiments on standard image denoising tasks.
These preliminary evaluations enable a controlled comparison of different methods and configurations.

For all variants, we follow the original DIP paper and set the maximum number of iterations to 2000.
For ES-based methods, we employ the ES-WMV criterion with a patience of 500 iterations.
For DIP-TV, we set $\lambda = 0.1$, and for SG-DIP, we set the number of different noise samples per iteration to 3, as recommended in the original paper.
For SGR-DIP, we linearly increase $\lambda$ from 1 to 10 throughout training. 
As discussed in Section~\ref{sec:architecture}, we use ECA in the skip connections because it tends to retain more fine-grained details, while yielding almost identical performance, as shown in Table~\ref{tab:ECA}. 

\begin{table}
    \small
    \centering
    \begin{tabular}{ l l c c }
        \toprule
        Skip Type &Method &PSNR (dB) $\uparrow$ &SSIM $\in [0,1]$ $\uparrow$ \\
        \midrule
        \multirow{2}{4em}{Conv} &SG-DIP (ES) &26.03 {\scriptsize (2.15)} &0.68 {\scriptsize (0.11)} \\
        &SGR-DIP &\underline{26.34} {\scriptsize (2.12)} &\textbf{0.70} {\scriptsize (0.11)} \\
        \midrule
        \multirow{2}{4em}{ECA} &SG-DIP (ES) &25.97 {\scriptsize (1.92)} &0.68 {\scriptsize (0.11)} \\
        &SGR-DIP &\textbf{26.37} {\scriptsize (2.04)} &\textbf{0.70} {\scriptsize (0.11)} \\
        \bottomrule
    \end{tabular}
    \caption{
        Comparison of different types of skip connections on the Set14 dataset.
        Noisy input images are generated at a PSNR of 15\,dB.
        Values represent mean and {\scriptsize (standard deviation)}.
    }\label{tab:ECA}
\end{table}

% TODO add DIP-TV (ES) & SG-DIP, fix order
\begin{figure}
    \centering
    \includegraphics[width=\textwidth]{img/fig_6.1_0.png}\\
    \vspace{20pt}
    \includegraphics[width=\textwidth]{img/fig_6.1_1.png}\\
    \vspace{20pt}
    \includegraphics[width=\textwidth]{img/fig_6.1_2.png}
    \caption{
        Visual comparison of various DIP variants on the CBSD68 dataset.
        Noisy input images are generated at a PSNR of 15\,dB.
    }\label{fig:CBSD68}
\end{figure}

% TODO add DIP-TV (ES) & SG-DIP 
\begin{table}
    \small
    \centering
    \begin{tabular}{ l c c c }
        \toprule
        Method &PSNR (dB) $\uparrow$ &SSIM $\in [0,1]$ $\uparrow$ &Runtime (m) $\downarrow$\\
        \midrule
        DIP &22.32 {\scriptsize (0.74)} &0.43 {\scriptsize (0.11)} &\underline{1.31} {\scriptsize (0.01)} \\
        DIP (ES) &25.64 {\scriptsize (2.07)} &0.62 {\scriptsize (0.07)} &\textbf{0.50} {\scriptsize (0.08)} \\
        DIP-TV &26.06 {\scriptsize (1.37)} &0.65 {\scriptsize (0.07)} &1.45 {\scriptsize (0.01)} \\
        DDIP &24.54 {\scriptsize (3.15)} &0.61 {\scriptsize (0.13)} &1.34 {\scriptsize (0.01)} \\
        SG-DIP (ES) &\underline{26.57} {\scriptsize (2.47)} &\underline{0.69} {\scriptsize (0.09)} &2.05 {\scriptsize (0.61)} \\
        SGR-DIP &\textbf{26.76} {\scriptsize (2.51)} &\textbf{0.70} {\scriptsize (0.10)} &3.34 {\scriptsize (0.09)} \\
        \bottomrule
    \end{tabular}
    \caption{
        Quantitative comparison of various DIP variants on the CBSD68 dataset (scaled to $256 \times 256$ pixels).
        Noisy input images are generated at a PSNR of 15\,dB.
        Values represent mean and {\scriptsize (standard deviation)}.
    }\label{tab:CBSD68}
\end{table}

Table~\ref{tab:CBSD68} presents the performance of the different DIP variants discussed in this work in terms of PSNR and SSIM for moderate noise.
Corresponding visual comparisons between the methods are available in~\ref{fig:CBSD68}.
Our method yields the best results; however, it is significantly slower than other approaches.

\begin{table}
    \small
    \centering
    \begin{tabular}{ l c c c c }
        \toprule
        &\multicolumn{2}{c}{20\,dB} &\multicolumn{2}{c}{10\,dB}\\
        \midrule
        Method &PSNR (dB) $\uparrow$ &SSIM $\in [0,1]$ $\uparrow$ &PSNR (dB) $\uparrow$ &SSIM $\in [0,1]$ $\uparrow$\\
        \midrule
        DIP &27.52 {\scriptsize (1.20)} &0.74 {\scriptsize (0.04)} &16.02 {\scriptsize (0.45)} &0.21 {\scriptsize (0.06)} \\
        DIP (ES) &27.10 {\scriptsize (1.72)} &0.73 {\scriptsize (0.09)} &22.52 {\scriptsize (1.58)} &0.48 {\scriptsize (0.05)} \\
        DIP-TV &28.29 {\scriptsize (2.31)} &\textbf{0.79} {\scriptsize (0.06)} &18.99 {\scriptsize (0.52)} &0.33 {\scriptsize (0.07)} \\
        DIP-TV (ES) &\underline{28.37} {\scriptsize (2.31)} &0.78 {\scriptsize (0.06)} &\underline{23.44} {\scriptsize (1.67)} &\textbf{0.56} {\scriptsize (0.08)} \\
        DDIP &23.15 {\scriptsize (2.84)} &0.58 {\scriptsize (0.14)} &22.58 {\scriptsize (2.57)} &0.54 {\scriptsize (0.10)} \\
        SG-DIP &\textbf{28.50} {\scriptsize (1.97)} &\textbf{0.79} {\scriptsize (0.07)} &12.34 {\scriptsize (1.16)} &0.12 {\scriptsize (0.05)} \\
        SG-DIP (ES) &27.42 {\scriptsize (2.40)} &0.75 {\scriptsize (0.09)} &\textbf{23.49} {\scriptsize (2.16)} &0.54 {\scriptsize (0.10)} \\
        SGR-DIP &26.99 {\scriptsize (2.36)} &0.73 {\scriptsize (0.13)} &23.29 {\scriptsize (2.48)} &\textbf{0.56} {\scriptsize (0.12)} \\
        \bottomrule
    \end{tabular}
    \caption{
        Quantitative comparison of various DIP variants on the Set14 dataset and different noise levels.
        Noisy input images are generated at PSNRs of 20\,dB and 10\,dB, respectively.
        Values represent mean and {\scriptsize (standard deviation)}.
    }\label{tab:noise-levels}
\end{table}

\begin{figure}
    \centering
    \includegraphics[width=\textwidth]{img/fig_6.2.png}
    \caption{
        PSNR curves
        % TODO add caption
    }\label{fig:noise-levels}
\end{figure}

\section{Distributed Acoustic Sensing Data}
