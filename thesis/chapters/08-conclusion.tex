\chapter{Conclusion}

In this work, we explored various DIP-based denoising methods and investigated their applicability to DAS data.
We introduced SGR-DIP, an extension of SG-DIP, and demonstrated its effectiveness on regular image data.
Furthermore, we showed that DIP-based methods can achieve good denoising performance on DAS data; however, not all approaches translate well to this context.
A key direction for future research is the development of an effective early stopping criterion for DAS data. This would not only enable the application of other DIP-based methods, which showed promising results in our image denoising experiments, but also reduce runtime, a major limitation of these approaches.
Additionally, exploring improved weight initialization strategies, e.g., through meta-learning~\cite{MetaDIP}, could further accelerate the optimization process.
