\chapter{Supplementary Material}

\renewcommand\addcontentsline[3]{}  % don't show appendix section in contents

\section{Acronyms}

\begin{table}[h!]
    \centering
    \begin{tabular}{ l l }
        \toprule
        \textbf{BM3D} & block-matching and 3D filtering \\
        \textbf{BN} & batch normalization \\
        \textbf{CNN} & convolutional neural network \\
        \textbf{DAS} & distributed acoustic sensing \\
        \textbf{DDIP} & deep diffusion image prior \\
        \textbf{DIP} & deep image prior \\
        \textbf{DL} & deep learning \\
        \textbf{ECA} & efficient channel attention \\
        \textbf{ES} & early stopping \\
        \textbf{ES-WMV} & early stopping via windowed moving variance \\
        \textbf{IDF} & integrated denoising framework \\
        \textbf{IP} & inverse problem \\
        \textbf{MSE} & mean squared error \\
        \textbf{N2N} & Noise2Noise \\
        \textbf{N2S} & Noise2Self \\
        \textbf{N2V} & Noise2Void \\
        \textbf{PSNR} & peak signal-to-noise ratio \\
        \textbf{SG-DIP} & self-guided deep image prior \\
        \textbf{SGR-DIP} & self-guided refinement deep image prior \\
        \textbf{SSIM} & structural similarity index measure \\
        \textbf{TV} & total variation \\
        \bottomrule
    \end{tabular}
    \caption{List of common acronyms.}
\end{table}

\section{DAS Experiment Details}\label{sec:DAS-details}

\begin{table}[h!]
    \centering
    \begin{tabular}{ l l }
        \toprule
        Dataset &URL \\
        \midrule
        SISSLE &{\url{https://datacommons.anu.edu.au/DataCommons/item/anudc:6317}} \\
        FORGE &{\url{https://constantine.seis.utah.edu/datasets.html}}\\
    	\bottomrule
    \end{tabular}
    \caption{Sources for the DAS datasets used in our experiments.}\label{tab:das-sources}
\end{table}

\begin{table}[h!]
    \centering
    \begin{tabular}{ l l c c c c }
        \toprule
        Sample & File name & Channels & \shortstack{Time\\span (s)} & \shortstack{Channel\\spacing (m)} & \shortstack{Sample\\rate (Hz)} \\
        \midrule
        \textit{FORGE 1} & {\scriptsize \shortstack[l]{FORGE\_78-32\_iDASv3-P11\_\\UTC190428070308.sgy}} & 200--1160 & 9.75--10.75 & 1 & 1000 \\
        \\[-1em]
        \textit{FORGE 2} & {\scriptsize \shortstack[l]{FORGE\_78-32\_iDASv3-P11\_\\UTC190423213209.sgy}} & 200--1160 & 0.75--1.75 & 1 & 1000 \\
        \\[-1em]
        \textit{FORGE 3} & {\scriptsize \shortstack[l]{FORGE\_78-32\_iDASv3-P11\_\\UTC190419001218.sgy}} & 50--1010 & 3.5--4.5 & 1 & 1000 \\
        \\[-1em]
        \textit{SISSLE 1} & {\scriptsize \shortstack[l]{south30\_50Hz\_UTC\_\\20230412\_074907.359.h5}} & 400--912 & 42--52.24 & 4 & 50 \\
        \bottomrule
    \end{tabular}
    \caption{Detailed information on the DAS samples used in our experiments.}\label{tab:sample-details}
\end{table}

\section{Derivation of the Noisier2Noise Assumption}

Let $x$ be a clean signal, and let $y = x + n$ be a noisy observation of this signal.
Further, let $z = y + m = x + n + m$ be an even noisier observation of the clean signal, with $m$ representing independent noise of the same distribution as $n$, i.e., $n\overset{d}{=}m$, leading to
\begin{equation}
    \E{n \mid z} = \E{m \mid z} = \E{\tfrac{n+m}{2} \mid z}.
\end{equation}
Using the linearity of expectation, we obtain:
\begin{equation}
\begin{split}
    \E{y \mid z} &= \E{x+n \mid z}\\
    &= \E{x \mid z} + \E{n \mid z}\\
    &= \E{x\mid z} + \E{\tfrac{n+m}{2} \mid z}.
\end{split}
\end{equation}
Therefore, our best estimate of $\E{y \mid z}$ is $x + \frac{n + m}{2}$.
